\chapter{Benutzerzentrierte Auswertung} %TODO

Das folgende Kapitel führt zunächst in die mögliche Bestimmung der Anforderungen der Nutzer:innen an eine graphisch-interaktive Entwicklungsumgebung ein. Daraufhin wird ein Konzept erläutert, wie eine solche Anwendung mit Hilfe von Nutzertests bewertet und verbessert werden kann. Zu Ende des Kapitels wird erläutert wie diese Konzepte umgesetzt und gezielt an die Zielgruppe verteilt wurden.

\section{Gestaltung der Anforderungserhebung} % Wie wurden anfänglich die Anforderungen der Nutzer erhoben? Wie wurden die Nutzer ausgewählt?

Damit die entstehende Entwicklungsumgebung genau auf die Bedürfnisse der Nutzer:innen angepasst ist, müssen anfänglich die Anforderungen der Nutzer abgefragt werden. Dazu bietet sich, besonders durch die Möglichkeit der Einbindung von zusätzlichen Medien, eine gezielte Online-Befragung an \cite{Schnell2018MethodenES}. Die Ergebnisse dieser Umfrage werden dann in einen Style-Guide für die Entwicklung der tatsächlichen Anwendung übersetzt.

Neben Fragen zu Gestaltungsentscheidungen, wie die Platzierung von zusätzlichen Informationen oder der Eingabeform von Parametern, ist es auch nützlich die Anwendung mit etablierten Programmierumgebungen im Bezug auf die Greifbarkeit und Interesse an der Nutzung zu vergleichen. Des Weiteren können noch Anforderungen der Nutzer:innen an die Anwendung abgefragt werden, um die Relevanz von verschiedenen Funktionen für die Nutzer:innen zu vergleichen und diese entsprechend bei der Entwicklung der Prototypen zu priorisieren. Abschließend ist es hilfreich relevante demographische Informationen der Teilnehmer:innen abzufragen.

Bei jeder der Fragen ist zu beachten, dass diese verständlich, knapp und neutral formuliert ist, sie nicht durch unnötige Verneinungen kompliziert werden, alle Teilnehmer:innen an der Umfrage die Frage beantworten können und ein Informationsgewinn durch die Beantwortung der Frage vorliegt \cite{Jacobsen2019PraxisbuchUuU}. Um technische Probleme zu vermeiden und die Umfrage möglichst barrierefrei auf verschiedensten Geräten und Browsern darstellen zu können, sollte eine Online-Umfrage-Applikation zu der Erstellung und Durchführung der Umfrage verwendet werden \cite{Schnell2018MethodenES}. Die verschiedenen Abschnitte des Fragebogens sollten jeweils auf einer eigenen Seite angezeigt werden, die ebenso den Fortschritt der Umfrage anzeigt. Diese einzelnen Seiten sollten alle nötigen Informationen beinhalten um die Fragen zu beantworten und ohne scrollen angezeigt werden können \cite{Jacobsen2019PraxisbuchUuU}.

\subsection{Vergleich mit bestehenden Anwendungen}
Ein Vergleich mit bestehenden Benutzeroberflächen ist nötig, um zu erfahren, ob der gewählte Ansatz einen Vorteil gegenüber bereits existierenden Lösungen darstellt. Besonders interessant ist ein Vergleich der Offenheit der Teilnehmer:innen an der Nutzung der verschiedenen Anwendungen.

Um einen unbefangenen Vergleich zwischen den Ansätzen zu ermöglichen, wird erst eine Interaktions-Aufgabe eingeführt, die der Roboter ausführen soll. Im Folgenden werden sowohl die etablierte Software, als auch das erste Mock-Up der Entwicklungsumgebung vorgestellt und erklärt. Abschließend wird die Implementierung der anfänglich eingeführten Aufgabe in beiden Anwendungen abgebildet.

Nach der Einführung in die Anwendungen werden mehrere Aussagen zu der Verständlichkeit und Interesse an der Nutzung einer solchen Benutzeroberfläche aufgeführt. Die Teilnehmer:innen an der Umfrage sollen dann ihre Übereinstimmung mit diesen Aussagen auf jeweils einer 5-Punkte Likert-Skala für alle vorgestellten Nutzeroberflächen beantworten. Abgefragte Einstellungen können Verständlichkeit der Benutzeroberfläche, Einschätzung der eigenen Fähigkeit die jeweilige Anwendung zu nutzen und weiteren Fragen zu spezifischen Teilen der Oberflächen beinhalten.

Die aus diesen Fragen entstehenden Daten können zu einer Einschätzung führen, wie nützlich der gewählte Ansatz ist und ob eine weitere Verfolgung dieses Ansatzes hilfreich ist, oder ein andres Konzept ausgearbeitet werden muss.

\subsection{Designentscheidungen} % TODO Besserer Name
Im darauf folgenden Abschnitt des Fragebogens werden verschiedene Funktionen des Roboters vorgestellt um Designentscheidungen am graphischen Interface der Entwicklungsumgebung zu treffen. Dazu werden jeweils mehrere Mock-Ups von Blöcken in Form von Bildern aufgeführt, die die beschriebene Funktion des Roboters umsetzen sollen. Die Teilnehmer:innen werden dazu aufgefordert, das Mock-Up auszuwählen, das am einfachsten zu verstehen ist. 

Die Auswahl der Optionen ist so zu treffen, dass die Ergebnisse jeweils Designentscheidungen repräsentieren. Beispielweise können die Optionen die Nutzung von Drop-Down-Menüs, Textfeldern oder anderweitiger Eingabe von Parametern unterscheiden. Die hier gesammelten Antworten werden in einen Style-Guide übersetzt, der die Vorlieben der Nutzer:innen repräsentiert.

\subsection{Funktionelle Anforderungen}
Eine weitere wichtige Komponente des Fragebogens ist die Abfrage der Anforderungen, die die Nutzer:innen an die Anwendung haben. Da diese inzwischen eine Einführung in die Entwicklungsumgebungen erhalten haben und in den Fragen zu den Designentscheidungen mit möglichen Mock-Ups der Kontrollelemente vertraut gemacht wurden, können sie hier einzelne Anforderungen einschätzen.

Dies wird ebenfalls in Form von Aussagen umgesetzt, die die Teilnehmer:innen mit einer 5-Punkte Likert-Skala beantworten. Die Aussagen sind jeweils als technische Anforderungen formuliert, die Nutzer:innen von solchen Anwendungen an diese stellen könnten. Beispielsweise kann abgefragt werden, in welcher Sprache die Steuerelemente angezeigt werden sollten, welche weiteren Funktionen in der Oberfläche integriert werden sollten und welche Anforderungen an die Ausführung der Anwendung gestellt werden.

Diese Ergebnisse werden dann teilweise, je nach Art der Aussage, in den Style-Guide oder eine Prioritäten-Liste der Anforderungen eingearbeitet, die die Wichtigkeit der Funktionen für die Prototypen und das endgültige Produkt angibt.

\subsection{Sonstige Datenerhebung}
In einem abschließenden Teil können noch Fragen gestellt werden, die Informationen über die Erfahrungen der Teilnehmer:innen mit Programmierung erheben. Ebenso können für die weiterlaufende Entwicklung relevante demographische Informationen abgefragt werden. Diese Abfrage kann in freier Form passieren.

Um Teilnehmer:innen für die darauf folgenden Nutzertests an den entstehenden Prototypen zu rekrutieren werden diese auf der letzten Seite der Umfrage dazu aufgefordert bei Interesse ihre E-Mail Adresse anzugeben. Da diese dann auch mit den Daten der vorherigen Seiten verknüpft sind, kann später analysiert werden, ob die Teilnehmenden an den Nutzertests repräsentativ für die Befragten stehen.

\section{Instanziierung der Testsettings} %TODO
Wie wurden die entstandenen Prototypen getestet? Welche Aufgaben sollten die Nutzer durchführen? Welche Involvierung hatte der Beobachter der Tests?

\section{Operationalisierung}
Welche Nutzer? Etc.