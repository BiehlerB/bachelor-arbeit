\chapter{Theoretische Grundlagen}

Alles was an theoretischen Grundlagen benötigt wird, sollte auch eher kurz gehalten werden.
Statt Grundlagenwissen zu präsentieren, eher auf die entsprechenden Lehrbücher verweisen.
Etwa: Tiefer gehende Informationen zur klassischen Mechanik entnehmen Sie bitte \cite{kuypers}.

Jede Untersektion sollte in etwa 1-3 Seiten machbar sein.

\section{Pepper}
Eine kurze Erklärung zu Pepper für Laien.

\section{Blockly}
Eine kurze Erklärung zu den Grundlagen von Blockly.
Evtl. auch kurz erklären, dass und wieso Blockly mit Kindern/Lernenden genutzt wird.

\section{Iterative Entwicklung}
Kurze Einführung in die Methodik der iterativen Softwareentwicklung.